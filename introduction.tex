\section{Introduction}
\label{sec.introduction}

%\cappos{Paragraph: Bad guys (hackers, NSA, etc.) are using zero day bugs in 
%the kernel to do bad things...}
%For many years system designers and developers have had to deal with a common security issue --
%how to defend against threats hidden within the system's own code. Despite
% decades of effort, researchers still frequently uncover new flaws in operating
% system kernels. Such flaws are dangerous because, if triggered by an untrusted
% program, these bugs can lead to such security threats as privilege escalation
% [CVE-2016-0728], [CVE-2015-8660], denial of service [CVE-2015-8539], [CVE-2015-5364],
%  and memory corruption [CVE-2014-9529].
%
The number of attacks that exploited zero-day vulnerabilities has more than 
doubled from 2014 to 2015~\cite{zero-day}. Skilled hackers can find a security 
hole in a system and use it to hold hostage of the system's users. Similarly, zero-day 
flaws can be exploited by government agencies~\cite{fbi-ff, fbi-iphone} and
lead to breach of privacy. \cappos{Must be OS kernel zero days.  Firefox
doesn't matter and privacy isn't the biggest concern.  It's someone breaking
into systems due to a flaw like this.}
%It is difficult to defend against these new and 
%unknown vulnerabilities, leading to an arms race between hackers and software 
%developers. 
%


In theory, running a program in an operating system virtual machine, such as 
VirtualBox, VMWare workstation, and Docker should prevent it from triggering 
zero-day bugs in the host OSes kernel.  However, to be 
effective, this isolation must function in two cases.  First, the OSVM's 
software must not contain bugs that allow the program to escape the 
OSVM's containment and interact directly with the host OS.  These
issues are very common in OSVMs (Virtualbox reports 40 such 
vulnerabilities~\cite{Virtualbox-Vulnerabilities} and more than 100 bugs have 
been found in VMware Workstation~[need cite]).  While it is understandable that
bugs occur in the large amount of complex code that is needed for such a 
system, these flaws present a major security risk to users [cite].
\cappos{revise ending?}
\yanyan{can we check if OSVM is a standard acronym?}


The other key situation where isolation is not effective is when a malicious 
program  can access a portion of the host OS's kernel that contains a 
zero-day flaw.  This typically occurs even while the containment of the OSVM is 
working as designed but without knowledge of a flaw in the OSVM.  The reason
this occurs is that many of the system calls made within an OSVM eventually 
result in system calls in the host OS (e.g., network I/O from the guest
OSVM results in network I/O by the host OS kernel).  If one of these paths in
the OS kernel contains a zero-day security bug, the malicious program
may be able to trigger it.

%aspect of this isolation is preventing the program from triggering bugs
%in the host OS kernel or in the OSVM itself.  However,

%The search for an ideal way to protect kernel code has been hampered by limited
%knowledge of the security impact of its interaction with user programs.
%A number of strategies have been attempted. One prevailing belief was that virtual
%machines could provide enough isolation to execute programs safely.
%However, when Virtualbox reports 40 vulnerabilities \cite{Virtualbox-Vulnerabilities},
% and more than 100 bugs have been found in VMware Workstation,
%it suggests that isolation alone may not be enough. Other researchers have looked
% to the development of metrics capable of pinpointing bugs, and,
%securing just those vulnerable areas\cite{PittSFIeld, ozment2006milk}. As these
%examples indicate, the design of secure systems has been somewhat “hit-or-miss.”
% Observations may lead to
% the reduction of specific threats, but not to a reproducible, broadly applicable,
%quantitative design solution.

With this paper, we move one step closer to designing OSVM systems that are
resilient to zero-day flaws.   We start
 with the proposition that kernel code found in popular paths, associated with frequently-used programs,
has less potential risk of bugs than code in less-used parts of the kernel.
We performed a quantitative analysis of resilience
  to flaws in two versions of the Linux kernel, and
found that about 3\% of the bugs were present in popular code paths,
despite these paths accounting for about one third of the total reachable 
kernel code.  
Therefore, OSVMs that only use these kernel paths will 
greatly increase resilience to zero-day bugs in the host OS kernel.
%When we ran the same study on the same Linux kernel versions 
%using two other metrics
%(Chou \cite{PittSFIeld} and Ozment \cite{ozment2006milk}),
%we found them less effective at predicting the location of zero-day bugs.

Guided by this knowledge, we propose a new design scheme for a secure virtual machine that
accesses only the popular code paths through a very small trusted computing base.
We name it \lip, as it locks kernel access to only code
associated with popular programs. %However, the popular kernel functionality is
%not sufficient to implement all of the system call functionality needed by 
%applications.  
However, there are many system calls outside of the popular paths that need to 
be supported to run applications~\cite{tsai2016study}.  Our second
observation in \lip is to re-create needed-but-risky OSVM functionalities
(e.g., POSIX) in a 
sandbox that only allows access to popular kernel paths.  In this way a bug
inside of the OSVM implementation does not result in a system compromise
so long as the minimal sandbox remains uncompromised.

%This is sufficient to
%perform operations for file systems, such as
%networking, threading, memory management, and other required functions.
%\cappos{The fact it is a PL sandbox is an implemenation detail, not design 
%detail...  Comment out the next sentence if you agree.}
%For complex functionalities,  \emph{Lock-in-Pop} re-creates system calls in
%a memory-safe programming language sandbox.

To demonstrate the viability of \emph{Lock-in-Pop} we use it to implement a 
prototype virtual machine that can offer enhanced security without sacrificing
 basic functionality. Dubbed Lind, it pairs two components -- Google's Native Client
(NaCl) and Seattle's Repy. NaCl serves as a computational module that isolates
binaries, providing memory
safety for legacy programs running in our OSVM. It also passes system calls
invoked by the program to the operating system interface, called SafePOSIX. 
SafePOSIX re-creates the broader POSIX functionalities needed by applications.
SafePOSIX does this while being contained within the Repy sandbox, which
provides an API that only allows access to popular kernel paths.  The small 
(8K LOC) sandbox kernel of Repy contains flaws in SafePOSIX 
to prevent them from allowing direct access to the host OS kernel.


\cappos{Aren't we about to say something like this in the contributions?}
To test Lind's effectiveness, we replicated 35 kernel bugs that were
discovered in the Linux kernel version 3.14.1.  We attempted to trigger those
bugs in seven other virtualized environments,
including VirtualBox, VMWare Workstation, Docker, LXC, KVM, QEMU, and Graphene.
Our results show that applications in Lind were substantially least likely to trigger 
kernel bugs. \yanyan{I think this is ok.}


In summary, the main contributions of this paper are as follows:
\cappos{I want to revisit this later on...}

\begin{itemize}\setlength\itemsep{0em}
%\item
%We postulate a new approach for securing privileged code,
%such as the OS kernel, based on the idea that popular kernel paths contain 
%fewer bugs.  \cappos{I'm not sure how this differs from 3...}

\item
We propose a quantitative metric that evaluates security at the line-of-code 
level.  We compare to other proposed metrics
such as the age of code~\cite{ozment2006milk}, and
the increased risk in driver code~\cite{PittSFIeld}.  We find that the metric of
choosing popular kernel paths is much more effective at identifying lines of
code that are unlikely to contain flaws.
%of the kernel that
%After testing this metric against others, we find it effective in locating
% bugs in the Linux kernel.

\item
Based on the metric, we postulate a new approach for securing privileged code, 
and develop a new design scheme called \emph{Lock-in-Pop}. It
accesses only popular code paths
through a very small trusted computer base.
The need for complex functionality is addressed by re-creating riskier systems calls
in a memory-safe programming language within a secure sandbox.

\item
We build a prototype virtual machine, Lind, using the \emph{Lock-in-Pop} design,
 and test its effectiveness against seven other security systems. We find that
Lind exposes 5-14x fewer kernel code paths with bugs than prior production and
academic systems.
\cappos{Yes, I know I'm framing this a little different from the abstract
and prior para in the intro.  I'm trying to find what works best...}
\end{itemize}

\cappos{I almost feel we need a diagram to `map' some parts of the paper.
Perhaps we should show something like Figure 4 (but see my comments) and
label where our analysis / description focuses.  Right now, there is enough
confusion between popular paths / sandboxed POSIX that it may make the paper
hard to follow.}

The remainder of this paper is organized as follows.
Section \ref{sec.motivation-and-background} presents the scope of our study 
and precisely describes our threat model.
%To understand which kernel paths a 
A study of earlier kernel protection metrics and the results of tests we ran to compare their performance
against our newly proposed design are discussed in Section \ref{sec.metric}.
We will discuss handling bugs in the OSVM software in 
Section~\ref{sec.design}, while focusing on preventing zero-day kernel
bugs in the host OS from being exploited in the meantime.
Section \ref{sec.design} also describes our \emph{Lock-in-Pop} 
design scheme.
In Section \ref{sec.implementation} we discuss the construction of the Lind 
prototype, while Section \ref{sec.evaluation} provides a quantitative security 
analysis that examines how often zero-day bugs are in reachable kernel paths
in different virtualization systems.
Section \ref{sec.limitation} outlines possible future initiatives.
Finally, Section \ref{sec.related_work} reviews existing design metrics, as well as 
techniques that share some of Lind's security goals,
while the cogent points relayed in the paper are reviewed in Section \ref{sec.conclusion}.
%\yiwen{Can we lose the last map paragraph? I'd like to remove it and it will save a lot of space.}
%\cappos{Don't worry about space yet.}
%
%The remainder of this paper is organized as follows.
%Section \ref{sec.motivation-and-background} presents the scope of our study and our threat model.
%A study of earlier kernel protection metrics and the results of tests we ran to compare their performance
%against our newly proposed design are discussed in Section \ref{sec.metric}.
%Section \ref{sec.design} describes the development of our \emph{Lock-in-Pop} design scheme.
%In Section \ref{sec.implementation} we discuss the construction of the Lind prototype,
%while Section \ref{sec.evaluation} provides the  results of tests that directly compare Lind's performance
%in preventing the triggering of bugs to other virtualization systems.
%Section \ref{sec.limitation} outlines possible future initiatives.
%Finally, Section \ref{sec.related_work} reviews existing design metrics, as well as techniques that share some of Lind's security techniques and goals,
%while the cogent points relayed in the paper are reviewed in Section \ref{sec.conclusion}.
