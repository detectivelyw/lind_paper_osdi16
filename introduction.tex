\section{Introduction}
\label{sec.introduction}

For many years system designers and developers have had to deal with a common security issue -- 
how to defend their programs from threats hidden within its own code. Despite the efforts of programmers 
to develop systems free of vulnerabilities, researchers frequently uncover new flaws in operating system kernels that 
can be triggered by interaction with untrusted programs. Yet, users want to be able to run any applications 
on open source operating systems, even if an application is untrusted. Complicating the problem is that 
a possible compromise of privileged code in the OS kernel is particularly dangerous because any bugs or flaws triggered there 
can pose a direct security threat to other parts of the system. Severe vulnerabilities have been discovered 
which resulted in serious security problems such as privilege escalation \cite{CVE-2016-0728, CVE-2015-8660}, 
denial of service \cite{CVE-2015-8539, CVE-2015-5364}, and memory corruption \cite{CVE-2014-9529}.

Solving this problem has been hampered by the fact that much is still unknown about the security impact of 
interaction between user programs and privileged code. As a result, the design of secure systems has been 
somewhat ``hit-or-miss.'' Many people believe that existing virtual machines provide enough isolation to 
execute programs safely. However, vulnerabilities found in those virtual machines suggest that there is still a long way to go. 
Virtualbox has 40 reported vulnerabilities \cite{Virtualbox-Vulnerabilities}, 
while more than 100 bugs have been found in VMware Workstation \cite{VMWare-Vulnerabilities}. 
Other studies have looked to the development of security metrics \cite{PittSFIeld, ozment2006milk} to pinpoint vulnerable areas. 
%A 2001 study by Chou, et al. \cite{PittSFIeld} provided evidence that more bugs could be found in device drivers than in other parts of the OS kernel. 
%Another study presented evidence that code grew safer with time, so newer code could contain more hidden bugs than older code \cite{ozment2006milk}. 
%Both of these studies, and many others, have also contributed to improved kernel safety technologies by analyzing where vulnerabilities are likely to exist. 
%However, despite all of this research, solutions that respond broadly to security concerns still remain elusive. 
%In other words, we are still lacking reproducible quantitative measures to evaluate the security of kernel code. 
However, we are still lacking reproducible quantitative measures to evaluate the security of kernel code. 

In this paper, we move one step closer to that goal by documenting the development and 
testing of a new design that offers a more reliable foundation for building secure virtual machines. 
Proposing that code found in popular paths, associated with frequently-used programs, 
has less potential risk than code in less-used parts of the kernel, we set out to determine 
which lines could be executed safely within the OS kernel. 
After running a quantitative analysis of resilience to flaws in two versions of the Linux kernel, 
we used this data to predict where zero-day bugs could later be discovered. 
We found that only 2.5\% - 4.0\% of the bugs were present in popular code paths, 
despite these paths accounting for nearly 32.2\% - 34.5\% of the total reachable kernel code. 
When we ran the same study using the aforementioned other metrics (Chou \cite{PittSFIeld} and Ozment \cite{ozment2006milk}), 
we found them less  effective at predicting the location of zero-day bugs in the Linux kernel versions that we tested.

With this knowledge, we propose a new design for a secure virtual machine that 
accesses only the popular code paths through a very small trusted computing base. 
This design scheme, named \emph{Lock-in-Pop}, locks out kernel access to all code except 
that found in paths associated with frequently-used popular programs. \emph{Lock-in-Pop} requires an interface 
that builds on only basic OS primitives, performing essential operations for file system, 
networking, threading, and memory management. 
To accommodate needs for complex functionality, it re-creates riskier system calls in a memory-safe programming language sandbox.

Based on the \emph{Lock-in-Pop} design scheme, we implement a new prototype virtual machine that 
builds complex operating system functionality (e.g., directories and permissions) utilizing only commonly-used primitives (e.g., reading and writing to a file). 
Dubbed Lind, the virtual machine pairs two components -- Google's Native Client (NaCl) and Seattle's Repy. 
NaCl serves as a computational module that isolates binaries, providing memory safety for a legacy program like Apache, 
while passing the system calls to the operating system interface. 
The operating system interface, called SafePOSIX, is isolated within the Repy sandbox, and ensures access to only popular paths. 
Complex operating system functionality in an isolated environment is provided by the small (8K LOC) Repy sandbox kernel. 
This provides straightforward access to the required popular kernel paths, 
while allowing more complex functionality to be built on top. 
In this manner, Lind can offer enhanced security without losing basic functionality. 

To test Lind's effectiveness, we replicated 35 kernel bugs in the Linux kernel version 3.14.1, 
and attempted to trigger them in seven other virtualized environments, 
including popular commercial systems such as VirtualBox, VMWare Workstation, 
Docker, LXC, KVM, QEMU, as well as the research systems Graphene and Lind. 
Our results show that applications in Lind were least likely to trigger kernel bugs, 
with only one out of the 35 (2.9\%) kernel vulnerabilities tested for being triggered.

In summary, the main contributions of this paper are as follows:

\begin{itemize}\setlength\itemsep{0em}
\item
We postulate a new approach for securing privileged code, 
such as that in the OS kernel, based on the idea that commonly-used popular kernel paths contain fewer bugs. 

\item
We propose a quantitative security metric to evaluate the security of kernel code at the line-of-code level. 
After testing our safety metric against others, we find it effective in locating where bugs are in the Linux kernel, 
which provides a more accurate base for developing new secure designs for protecting the OS kernel.

\item
We develop a new design scheme called \emph{Lock-in-Pop} that accesses only popular code paths 
through a very small trusted computer base. 
The need for complex functionality is addressed by adding a strategy to the \emph{Lock-in-Pop} design that 
uses a memory-safe programming language to recreate riskier systems calls within a secure sandbox. 

\item
We build a prototype virtual machine called Lind and tested its effectiveness at not triggering known zero-day bugs 
against seven other security systems. We found that Lind triggers only one (2.9\%) of the 35 zero day bugs we had targeted, 
making it an order of magnitude more secure than other systems.
\end{itemize}

\yiwen{Can we lose the last map paragraph? I'd like to remove it and it will save a lot of space.}

The remainder of this paper is organized as follows.
Section \ref{sec.motivation-and-background} presents the scope of our study and our threat model. 
A study of earlier kernel protection metrics and the results of tests we ran to compare their performance 
against our newly proposed design are discussed in Section \ref{sec.metric}. 
Section \ref{sec.design} describes the development of our \emph{Lock-in-Pop} design scheme. 
In Section \ref{sec.implementation} we discuss the construction of the Lind prototype, 
while Section \ref{sec.evaluation} provides the  results of tests that directly compare Lind's performance 
in preventing the triggering of bugs to other virtualization systems. 
Section \ref{sec.limitation} outlines possible future initiatives. 
Finally, Section \ref{sec.related_work} reviews existing design metrics, as well as techniques that share some of Lind's security techniques and goals, 
while the cogent points relayed in the paper are reviewed in Section \ref{sec.conclusion}.