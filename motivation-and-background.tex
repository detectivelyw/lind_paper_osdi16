\section{Goals and Threat Model}
\label{sec.motivation-and-background}

\textbf{Goals.}
Our goal is to design and build secure virtualization systems, that allow untrusted programs to run
on an unpatched and vulnerable host OS (Linux OS in this study), without causing damages to other parts of the system.
\yanyan{what are the other parts of the system?}
We focus on providing a solution when a host OS contains vulnerabilities
that an attacker could exploit. To combat this, untrusted programs are executed in a secure virtualization system,
such as a guest OS VM, a system call interposition module, or a library OS. We are trying to design such a secure virtualization system
that could protect the underlying host OS and all the programs running in the host OS, when facing malicious attackers.

In the following, we define the level of security that we intend to enforce, and the possible attacks that we desire to protect against.
We assume that an attacker is able to run an attack program in our secure virtualization system, attempting to
exploit an unpatched vulnerability in the host OS. Our goal is to prevent the attack program from
triggering a vulnerability in the host OS.
To trigger a vulnerability, the attacker could have two approaches.
First, the attack program could exploit a flaw in the host OS by
performing an action in the secure virtualization system that triggers a flaw in
the host OS. Effectively, the attacker is taking advantage of the secure
system utilizing paths in the host OS kernel that have a vulnerability.
Second, the attack program could exploit a flaw in the secure virtualization system to
escape its containment. \yanyan{maybe explain what is to escape its containment?}
Once the attacker has exploited this type of flaws, the
attack program will be able to run arbitrary code, and therefore
can make system calls in the host OS directly, which allows
the attacker to trigger a known exploit. We try to prevent both attacks.

It should be noted that several solutions exist for building a virtualization system for running untrusted programs.
However, the following techniques are out of scope as they have different hardware and software requirements.
We do not compare with solutions that do not run on a full-fledged
privileged operating system, such as virtualization system that uses a
bare-metal hypervisor~\cite{Xen-03} or hardware-based virtualization
solution~\cite{IntelVT}. These solutions have substantial dependency on the underlying hardware, which is
different from our goal. In addition, they do not interact with a privileged OS kernel, which makes it very difficult
to directly compare with our solution that accesses and relies on the OS kernel.
\yanyan{I feel this make it sound like "accesses and relies on the OS kernel"
constrains our method.}

\noindent\textbf{Threat model.}
To summarize, our threat model makes the following assumptions.

\begin{itemize}\setlength\itemsep{0em}

\item The attacker possesses knowledge about one or more unpatched vulnerabilities in the host OS.

\item The attacker is permitted to execute any code in the secure virtualization system.

\item If the attack program can trigger a vulnerability in any privileged code,
whether in the host OS or the secure virtualization system, the attacker is then able to
compromise the system.

\end{itemize}

%Based on our threat model,
To achieve our goals, we need to have a better understanding about how vulnerabilities are
triggered by attack programs. It is therefore critical to know where vulnerabilities are located in
an OS kernel. Hence, we design a security metric that measures how bugs and vulnerabilities
are distributed in the kernel.
%We present and discuss about it in the following section.
