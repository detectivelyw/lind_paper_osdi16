\section{Goals and Threat Model}
\label{sec.motivation-and-background}

\textbf{Goals.}
Our goal is to design and build a secure virtualization system that allows
untrusted programs to run on an unpatched and vulnerable host OS (Linux OS in
this study), without triggering vulnerabilities that attackers could exploit.
Developing effective defenses for the host OS kernel is essential as kernel code
can expose privileged access to attackers that could lead to a system take-over.

To combat this threat, untrusted programs are often executed in a secure
virtualization system, such as a guest OS virtual machine, a system call interposition
module, or a library OS system. Our intent is to
build such a system capable of protecting not only
an underlying host OS, but also any programs running in it, from both malicious attacks
and accidental triggering by applications.
In this section, we define the scope of our efforts. We also briefly note why we
chose to exclude a few existing design schemes from this study.

We start by acknowledging that it is possible for an attack attempt to be staged
on a host OS in a secure virtualization system.
Furthermore, we anticipate that such an exploit could be done directly or indirectly.
In a direct exploit, the attacker
takes advantage of paths in the system's host OS kernel
that have a vulnerability. In an indirect exploit,
the attacker takes advantage of the exploited vulnerability to run arbitrary code, and
to make system calls in the host OS.
The secure virtualization system design we will propose
in Section 4 is designed to prevent both types of attacks.

\noindent
\textbf{Threat model.}
Based on the goals mentioned above, we make the following assumptions about the
potential threats our system could face:

\begin{itemize}\setlength\itemsep{0em}

\item The attacker possesses the knowledge about one or more unpatched vulnerabilities in the host OS.

\item The attacker is permitted to execute any code in the secure virtualization system.

\item If the attack program can trigger a vulnerability in any privileged code,
whether in the host OS or the secure virtualization system, the attacker is then considered successful
in compromising the system.

\end{itemize}

\noindent
\textbf{Exclusion.}
It should be noted that our study intentionally excludes several existing
approaches to virtualization systems. We considered these techniques
out of scope due to differences in hardware and software requirements.
Primarily we chose to exclude solutions that do not run on top of on a full-fledged
privileged operating system, such as a virtualization system that uses a
bare-metal hypervisor~\cite{Xen-03} or hardware-based virtualization
solution~\cite{IntelVT}.
These devices are dependent on the underlying hardware, and do not interact
with a privileged OS kernel. Such differences in both structure and function
make it difficult to directly compare these techniques to our proposed model.

Once we establish the scope of our study, it is clear we need a better understanding about how
bugs are triggered by attack programs. It is therefore critical to know where
vulnerabilities are located in an OS kernel. Our first step is to design
a security metric that can quantitatively measure how bugs and vulnerabilities
are distributed within the kernel.
