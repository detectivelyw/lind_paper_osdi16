\section{Goals and Threat Model}
\label{sec.motivation-and-background}

Our goal in this work is to design and build secure virtualization systems, that allow untrusted programs to run 
on an unpatched and vulnerable host OS (Linux OS in this study), without causing damages to other parts of the system. 
We focus on providing a solution on situations where there is a host OS that contains vulnerabilities
that an attacker could exploit. To combat this, untrusted programs are executed in a security virtualization system, 
such as a guest OS VM, system call interposition module, or library OS. We are trying to design such a security virtualization system 
that could protect the underlying host OS as well as all the programs that are running on the host OS, in the face of malicious attackers. 

We now define the level of security that we intend to enforce, and the possible attacks that we desire to protect against with our solution. 
In our study, we assume that the attacker is able to run an attack program in our security virtualization system with the goal of 
exploiting an unpatched vulnerability in the host OS. Our desired security goal is to prevent the attack program from successfully 
triggering an vulnerability in the host OS. 
To trigger and exploit an vulnerability, an attacker could potentially take two approaches. 
First, the attack program could exploit a flaw in the host OS by 
performing an action in the security virtualization system that triggers a flaw in
the host OS. Effectively, the attacker is taking advantage of the security 
system utilizing paths in the host OS kernel which have vulnerabilities.
Second, the attack program could exploit a flaw in the security virtualization system to 
escape its containment. Once the attacker has exploited this type of flaws, the
attack program will be able to run arbitrary code, and therefore 
can make system calls in the host OS directly, which allows
the attacker to trigger a known exploit. We try to prevent both of these attacking 
strategies in this work. 

It should be noted that several solutions exist for building a virtualization system for running untrusted programs. 
But the following ones are out of scope of this work because they have different hardware and software requirements.  
We do not compare with solutions that do not run on a full-fledged 
privileged operating system, such as virtualization system that uses a 
bare-metal hypervisor~\cite{Xen-03} or hardware-based virtualization 
solution~\cite{IntelVT}. Those solutions have substantial dependency on the underlying hardware, which is essentially 
a different approach than our proposal. In addition, they do not interact with a privileged OS kernel, which makes it very difficult 
to conduct direct comparison with our solution that accesses and relies on the OS kernel. 

To summarize, our threat model makes the following assumptions.

\begin{itemize}\setlength\itemsep{0em}

\item The attacker possesses knowledge about one or more unpatched vulnerabilities in the host OS.

\item The attacker is permitted to execute any code in the security virtualization system.

\item If the attack program can trigger a vulnerability in any privileged code,
whether in the host OS or the security virtualization system, the attacker is considered successful in 
compromising the system.

\end{itemize}

Based on our threat model, to achieve our goals, we need to have better understanding about how vulnerabilities can be accessed 
and triggered by attack programs. The question of where bugs are located in the OS kernel then becomes critical to our solution, 
which leads us to discover our security metric. We present and discuss about it in the following section. 