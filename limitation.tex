\section{Discussion}
\label{sec.limitation}

One of our challenges in conducting this study was deciding where to place the
limits of its scope. The literature documents a variety of strategies that, over
the years, have been used to secure the OS kernel. To explore any one strategy
in depth, we felt it was necessary to intentionally exclude consideration of
other valid approaches. In this section, we acknowledge some of the alternative
approaches that we chose not to investigate, as well as a few issues that we plan
to address in the future.

\textbf{Exclusions.}
In designing our study, we intentionally decided that we would not include any
discussion of native or “bare-metal” hypervisors that run directly on server
hardware. Though these devices, also called Type I hypervisors, have been used
for decades, we chose to exclude them because our initial emphasis in designing
Lind was to come up with something small and light that did not require any
alterations to the kernel. As we continue to develop our Lind prototype, we may
choose to study current examples of these native hypervisors to either directly
compare security and performance attributes, or to see if our design metric could
be adapted to these machines.

A second exclusion stems from our criteria for locating bugs. At the beginning
of our study, we identified a set of common but seriously dangerous zero day bugs
and then went looking for them within our obtained kernel traces. By looking only
for a specific subset of bugs, we acknowledged that we might be limiting our
ability to find a broader spectrum of kernel vulnerabilities. For example, bugs
caused by a race condition, or that involve defects in the internal kernel data
structures, or that require complex triggering conditions across multiple kernel
paths, may not be immediately found using our metric. As we continue to refine
our metric, we will look to also evolve our evaluation
criteria to allow us to find and protect against more complex brands of bugs.
In the meantime, we feel that avoiding the triggering of this initial set of bugs
through the use of our "Lock-in-Pop" design scheme can still address the security
needs for a significant segment of users.

\textbf{Future work.}
While our experiments were limited to the Linux kernel 3.14.1, our future work
will include testing its applicability to other operating systems, such as
Windows and Mac OS. Since the Lock-in-Pop design scheme is not dependent on the use of any
specific hardware, we believe it has the flexibility to be adapted to these other
widely-used systems.

The other challenge facing wide-scale adoption of Lind is improving its
performance in terms of bandwidth and other overhead factors. As addressed in
Section 6.4, Lind does incur some performance overhead. Since our initial tests
were focused on security issues, we did not take any steps to optimize
its performance before running these tests. Future work will focus on identifying
the factors that contribute to this overhead, and the best ways to make Lock-in-Pop
a cost-effective alternative to other virtual machine designs.

%However, not all bugs can be accurately checked in this manner.
Due to the nature of bugs, our proposed metric cannot assure whether a portion of code
is an absolute safe or risky area in kernel.
%So we are just trying to reduce the chances of triggering underlying kernel bugs, but can
%not promise to avoid all bugs.
For example, bugs that are caused
by a race condition cannot be identified by directly checking if certain lines of
code have been executed. For complicated bugs that involve defects in the internal
kernel data structures, or require complex triggering conditions across multiple
kernel paths, our metric will not be accurate.
%determine whether or not those bugs have been triggered.
In these cases, more complex metrics might be needed.

%While Lind executes programs like Apache and Tor, it does not
%support every system call or every possible set of arguments. For example,
%symbolic links are not supported.  While the SafePOSIX implementation could
%be extended to do so, we leave this for future work. Also, as mentioned in
%the previous section,


%There were also some avenues we intentionally excluded in this initial
%study that could form the basis for interesting research projects in the
%future. First, we chose not to explore bugs within the applications
%themselves.
%\cappos{move to motivation / threat model.}

Lind has not been optimized for performance.
%so the results we present here should be taken as a baseline.
%of what is possible.
We would like to explore what existing OS VM optimizations can be safely applied
and their impact.

We would like to test our metric in other operating systems, such as Windows and Mac OS.
Our experiments were limited to Linux kernel 3.14.1 and some of the typical virtualization systems that existed in Linux.
It would be interesting
%and beneficial to the advance of secure systems
if similar tests could be run in other widely-used operating systems.
%In particular, it would be interesting to see if
%having the host and guest VM run different operating systems would produce different security impacts.
