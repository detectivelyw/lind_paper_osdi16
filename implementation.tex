\section{Implementation of Lind}
\label{sec.implementation}

To test the feasibility of our Lock-in-Pop design in
Section \ref{sec.design}, we used it to implement a secure virtual machine
called Lind\footnote{\scriptsize An old English word for a shield constructed from two layers of
linden wood, providing strength in a lightweight form, the name is appropriate for
a virtual machine that adapts two technologies—\textendash
%Google's Native Client (NaCl) and Seattle's Repy, and that is lightweight in terms of
%LOC, 
yet still protects vulnerable OS kernel code from exploitation by
untrusted user programs.}. 
%The dual-sandbox design also minimizes overall overhead
%because the sandbox only incurs these cost when there is a system call. Lastly,
%Lind uses a native interface for execution, allowing CPU-and-memory-intensive
%applications to run at speeds that are equivalent to NaCl and near native speed.
%
As in Section \ref{sec.design}, the Lock-in-Pop design utilizes both a
computational module and a library OS module. In Lind, Native Client (NaCl) serves as the former,
while Restricted Python (Repy), and a POSIX interface built into it, perform the task of the latter.
Below we present a brief description of these components and how they were integrated
into Lind, followed by a brief example of how the system works.

\subsection{Primary Components}

\textbf{Native Client.}
We use NaCl to isolate the computation of the user application
from the kernel. NaCl allows Lind to work on most types of legacy code.
It compiles the programs to produce a binary with software fault isolation.
This prevents the majority of the application from performing system calls
or executing arbitrary instructions.
%
To perform a system call, the application will call into a small privileged
part of the NaCl TCB that forwards system calls, usually to the OS for
processing. To build Lind, we changed the NaCl TCP to
forward these calls to the library OS that we call SafePOSIX (details below)
for processing.
% the following doesn't say anything new
%The NaCl glibc module contains stubs that reject operations
%that Chrome would not handle.  We added functionality to NaCl's glibc so that
%those calls could be forwarded into SafePOSIX for processing.

%NaCl can perform the functions required by the computation module well, and it is easy to
%connect with our system API module because NaCl uses glibc to perform system calls.
%Modification to NaCl's glibc would allow us to redirect those system call requests to our own system API module.
%
%NaCl is a sandbox used to execute untrusted x86 native code.
%It aims to give applications the computational performance of native applications without compromising safety.
%NaCl uses software fault isolation and a secure runtime to direct system interaction and
%side effects through interfaces managed by the program. It provides operating system portability
%for binary code while supporting performance-oriented features, such as thread support,
%instruction set extensions, such as SSE, and use of compiler intrinsics and a hand-coded assembler.
%It also allows the efficient execution of legacy code in the form of x86 and ARM binaries
%that are built with a lightly modified compiler tool chain.

\textbf{Seattle's Repy.}
To build an API to access the safe parts of the underlying kernel, we need
two things. First, we need a restricted sandbox that isolates computation
and only allows access to commonly used kernel paths.  We used
Seattle's Repy~\cite{Repy-10} sandbox to perform this task.
Second, to serve needs of the user programs, there has to be an operating system interface that 
provide system functions. We chose the world-wide accepted standard POSIX interface. 
And to enforce isolation, we need to build our POSIX implementation to run within that sandbox.
%pplications. For Lind, we used Repy to build our system API module.
%To be more specific, our system API module has a very small sandbox kernel
%as TCB,  written with Python. On top of the sandbox kernel,
%we use Repy code to safely reimplement complex system functions.

\textit{The Repy Sandbox Kernel.}
As the only piece of code in contact with the system call paths of the TCB,
the sandbox kernel's security is of paramount concern.
%The sandbox kernel needs to be secure and bug-free.
%Because it is the TCB of the system, any bugs in it could cause fatal problems.
%and allow attackers to access the OS kernel and gain kernel privilege.
We used Seattle's Repy system API due to its tiny sandbox kernel
(comprised of around 8K LOC), and its ability to provide straightforward
access to the minimal set of the system call API needed to build general
computational functionality. Repy allows
%fit our key design principle that our proposed system should
access only to the safe portions of the OS kernel with 33 basic API
functions, including 13 network functions, 6 file functions, 6 threading functions,
and 8 miscellaneous functions~\cite{Repy-10, RepyKernel}. The code is
written using style guidelines designed to ease security auditing
 of the code~\cite{style}. Most of these functions are simple and
regularly used system calls that access the commonly used kernel paths.

The Repy kernel code provides a solid foundation for our secure virtual
machine. It has been audited by a professional penetration tester and, since 2010,
there has also been a bug bounty program for security flaws in the sandbox.
The code is deployed in daily use across thousands of devices,
including on the Seattle testbed \cite{seattle}, and has been examined by
hundreds of parties. To date no security flaws have been found in the sandbox
kernel. Having a small, easily auditable piece of code thus helps to reduce the
risk of such occurrence.

However, Lind's security does not rely solely on Repy's
safety record. A reconfigured POSIX interface, and the isolation of
the NaCl module within the dual sandbox design enhance the protection it can
provide.
%Developers have reported
%XXX issues for problems in other parts of the systems. However,
%This does not provide any strong guarantees that bugs do not exist, and if
%they do, the security of the system could be compromised.
%However,

\subsection{Enhanced Safety in Call Handling}

The kernel interface is extremely rich and hard to protect.
The dual sandbox Lock-in-Pop design used to build Lind provides enhanced
safety protection through both isolation and a POSIX interface that
reformulates risky system calls to
provide sufficient API for legacy applications, with minimal impact on the kernel.

\textbf{An example system call execution.}
In Lind, a system call issued from user code are
received by NaCl, and then redirected to our system API module, which
includes a POSIX API to serve those requests. A POSIX API is a set of standard
operating system interfaces that provide
necessary operating functionality. A standard POSIX API is large and complex
enough to make it difficult to ensure its implementation is secure and bug-free.
Lind takes advantage of the fact that Repy is a programming language sandbox to
construct a variation on the POSIX API. Following the
Lock-in-Pop design, to service a system call in NaCl, a server routine in
Lind marshals its arguments into a text string, and sends the call and the arguments
to the Repy sandbox, where the "reimplemented" system call, marshals the result and
returns it back to NaCl. Eventually, the result is returned as the appropriate
native type to the calling program.

In the Lock-in-pop scheme, the file system API only need
to provide functionality of writing data to storage. \yanyan{why only discuss file system?
how about network, threading, etc?}
It eliminates the need of having a direct abstraction, the
concept of file permissions, links, or even the concept of multiple files.
%\lois{I moved this copy from Section 4. Please check that it makes sense where placed.
 %Also, as Yanyan notes, there should be an actual example here.}
The system API safe reimplementation is a set of more complicated system calls
derived from functions in the sandbox kernel.
We reimplement those system calls because we do not want this potentially risky user code
to have direct access to the underlying OS kernel.
Instead, our reimplementation layer serves as a mediator between the user code
and the OS kernel. The reimplementation is safe
because the reconstructed calls are isolated in a sandbox, and the code for the
reimplementation is written in a memory-safe programming language.

Here is an example of how this reimplementation would work with the symbolic link function.
If there is a bug in this function, rather than rely on the kernel code paths
for symbolic links, Lind will try to implement the incorrect behavior, such as creating a symbolic link 
with a deleted file. 
This denies the code the privileged access to the system the OS kernel does.
As a result, instead of creating a security issue, the application be denied access
to the file system.

Our design could include more than two sandboxes, e.g., by sandboxing the sandbox
kernel. However, in a secure system,
the lowest level sandbox eventually must have some fundamental,
even if limited, access to system resources, such as memory, and storage, threads.
Even if we were to sandbox the sandbox kernel and have additional sandboxes,
the one at the bottom level will still access the OS kernel in a similar way.
Thus, having multiple sandboxes does not provide any extra security benefits.

%Untrusted programs are run in NaCl,
%but access to all system resources is diverted to a Repy program.\yanyan{SafePOSIX?}
%This program is responsible for accessing the system on behalf of the Lind library
% OS. A NaCl sandbox is built on top of the Repy sandbox.
%

%
%Our choice to use Repy helped us solve this architectural security problem.

%
%As described in Figure \ref{fig:design} and Section 4.2, our design has
%two main components\textendash a computation module that isolates the
%application and a library OS module that isolates the complex
%portions of POSIX, a standard OS interface Lind provides. We choose
%Google's Native Client (NaCl)~\cite{NaCl-09}
%Seattle's Repy~\cite{Repy-10} as the library OS.
%(Figure \ref{fig:architecture}), since it is a restricted subset of Python,
%that works as a sandbox to provide a safer environment to run untrusted code.

%\begin{figure}%[h]
%\centering
%	\includegraphics[width=1.0\columnwidth]{diagram/lind_architecture_new.png}
%	\caption{Architecture of Lind including various components such as NaCl, NaCl glibc, and Repy Sandbox.
%	User level applications will issue system calls that are dispatched through the Repy OS connector that bridges the Lind system to the OS Kernel.}
%\label{fig:architecture}
%\end{figure}



\
%\begin{table}
%\centering
%\caption {Repy sandbox kernel capabilities that supports NaCl functions, such as networking, file I/O operations and threading.}
%
%  \begin{tabular}{ | p{2.5cm} | p{4.5cm} |}
%  \hline
%  \textbf{Repy Function} & \textbf{Available System Calls}  \\ \hline
%
%Networking & \emph{gethostbyname, openconnection, getmyip, socket.send, socket.receive, socket.close,
%listenforconnection, tcpserversocket.getconnection, tcpserversocket.close, sendmessage, listenformessage,
%udpserversocket.getmessage, and udpserversocket.close.} \\ \hline
%
%I/O Operations & \emph{openfile, file.close, file.readat, file.writeat, listfiles, and removefile.} \\ \hline
%
%Threading & \emph{createlock, sleep, lock.acquire, lock.release, createthread, and getthreadname.} \\ \hline
%
%Miscellaneous Functions & \emph{getruntime, randombytes, log, exitall, createvirtualnamespace,
%virtualnamespace.evaluate, getresources, and getlasterror.}  \\ \hline
%    \end{tabular}
%    \label{table:RepyKernel}
%\end{table}

%\begin{table}
%\centering
%\scriptsize
%\caption {System Functions in the Repy Sandbox Kernel.  \cappos{Need to
%clearly explain the takeaway.}}
%\begin{tabular}{|l|}
%  \hline
% \textbf{Network Functions} \\
%  \hline
%  gethostbyname(name) \\
%  \hline
%  getmyip() \\
%  \hline
%  openconnection(destip, destport, localip, localport, timeout) \\
%  \hline
%  socket.close() \\
%  \hline
%  socket.recv(numbytes) \\
%  \hline
%  socket.send(message) \\
%  \hline
%  listenforconnection(localip, localport) \\
%  \hline
%  tcpserversocket.getconnection() \\
%  \hline
%  tcpserversocket.close()\\
%  \hline
%  sendmessage(destip, destport, message, localip, localport) \\
%  \hline
%  listenformessage(localip, localport) \\
%  \hline
%  udpserversocket.getmessage() \\
%  \hline
%  udpserversocket.close() \\
%  \hline \hline
%  \textbf{File Functions} \\
%  \hline
%  openfile(filename, create) \\
%  \hline
%  file.close() \\
%  \hline
%  file.readat(sizelimit, offset) \\
%  \hline
% file.writeat(data, offset) \\
%  \hline
%  listfiles() \\
%  \hline
%  removefile(filename) \\
%  \hline \hline
%  \textbf{Threading Functions} \\
%  \hline
%  createlock() \\
%  \hline
%  lock.acquire(blocking) \\
%  \hline
%  lock.release() \\
%  \hline
%  createthread(function) \\
 % \hline
 % sleep(seconds) \\
  %\hline
  %getthreadname() \\
  %\hline \hline
  %\textbf{Miscellaneous Functions} \\
  %\hline
 % getruntime() \\
  %\hline
 % randombytes() \\
  %\hline
  %log(*args) \\
  %\hline
  %exitall() \\
  %\hline
  %createvirtualnamespace(code, name) \\
  %\hline
  %virtualnamespace.evaluate(context) \\
  %\hline
  %getresources() \\
  %\hline
  %getlasterror() \\
  %\hline
%\end{tabular}
%\label{table:RepyKernel}
%\end{table}



%Lind is designed to minimize the need to modify either sandbox. This is possible
%because the TCB of both were extremely small, and because the Lind code is run
%in both.
%\lois{Another odd sentence Yiwen and I discussed. I thought I knew how to
%fix it, but it still does not make sense. I would delete it.}

%The only complex part of Lind is the library OS, which runs in Repy.
%However, because Python is a very powerful language that provides rich functions,
%it helps to make the construction of Lind easier. The downside of Python is that
%some consider the language ``slow'',
%because it is dynamically typed rather than statically typed, it is interpreted rather than compiled,
%and Python's object model can lead to inefficient memory access.
%Nevertheless, the internals of an application in Lind are run in NaCl, a very high performance
%environment.

%This balances the performance of the system, with the ease of implementation and maintenance of the library OS component of Lind.
%\yanyan{I think this paragraph can be cut.}

%Lind is portable.
%Programs running inside Lind are using a standard POSIX glibc interface,
%and, since the Lind runtime is strictly user-level, it can work on many different platforms
%including Linux, Mac OS X and Windows.