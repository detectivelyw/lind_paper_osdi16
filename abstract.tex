\subsection*{Abstract}
A possible compromise of privileged code, such as that in the OS kernel, is a
daunting security threat. Severe vulnerabilities discovered in this
have resulted in privilege escalation, denial of service, memory corruption and
 other problems. However, securing this code has been hampered
  by a lack of knowledge about the location of these flaws. In this paper, we
 document the development of a new virtual
 machine to secure the OS kernel. We begin by proposing and testing a new
 quantitative evaluation metric that predicts code located
  in popular paths will contain fewer bugs. We leverage this knowledge
  for a design called  Lock-in-Pop, which locks out application access
 to all kernel code except that found in paths associated with frequently-used
 popular programs. Lastly, we build and test a secure virtual machine called Lind,
  which accesses only the popular code paths through a very small trusted computing
   base. Complex functionalities in Lind are addressed by using a memory-safe
    programming language within a sandbox to re-create riskier system calls.
  When Lind was tested against seven other virtual machines to see whether it would
   trigger any of 35 bugs examined in Linux kernel version 3.14.1, the threat of
    an attack was reduced to less than 3\%. This result is about an order of
    magnitude better than existing systems like VirtualBox (40\%), VMWare Workstation
     (31\%), Docker (23\%), LXC (34\%), QEMU (14\%), KVM (14\%), and Graphene (23\%).
