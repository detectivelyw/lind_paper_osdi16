\subsection*{Abstract}
%System designers and developers want users to be able to
%run applications -- even untrusted ones -- on open source operating systems
%without triggering potential security breaches. Yet, a possible compromise of privileged code
%in the OS kernel is particularly dangerous because bugs or flaws triggered there pose a direct security threat
%to the system itself and all other programs running in it.

Running untrusted programs on a vulnerable operating system kernel is a challenge.
%A number of security strategies have evolved over the years in response to these concerns,
A number of security strategies have evolved over the years to solve this problem,
such as system call interposition, operating system virtual machines, and library OSes.
However, without a widely applicable quantitative metric to guide efforts, systems
for securing privileged code
have emerged on almost a ``case-by-case'' basis. To develop more consistently
effective security systems,
more needs to be known about the location of flaws in kernel code.

In this paper, we document the development, testing, and application of a new
design for a secure virtual machine
that is based on a more accurate determination of what lines of code can be
executed safely within the Linux OS kernel.
With this knowledge, we devise a security metric called \emph{Lock-in-Pop},
as it locks out application access to all kernel code
except that found in paths frequently used by popular programs that we discover
contain fewer vulnerabilities.
Using this new design scheme, we build a secure virtual machine called Lind that accesses only
the popular code paths through a very small trusted computing base. Complex functionality is addressed
by using a memory-safe programming language to recreate riskier system calls in a sandbox.
Our test results for 35 bugs examined in Linux kernel version 3.14.1 show that Lind can
reduce the threat of an attack to less than 3\%.
This result is about an order of magnitude better than existing systems like VirtualBox (40\%), VMWare Workstation (31\%), Docker (23\%),
LXC (34\%), QEMU (14\%), KVM (14\%), and Graphene (23\%).
