\subsection*{Abstract}
%A possible compromise of privileged code, such as that in the OS kernel, is a
%daunting security threat. Severe vulnerabilities discovered in this code
%have resulted in privilege escalation, denial of service, memory corruption and
%other complications.
Virtual machines are widely used in practice, in part for their ability to
isolate potentially untrusted code from the rest of a system.
%Virtual machines have been developed to protect kernel code
%through isolation.
However, it is often possible to trigger zero-day flaws
in the host OS from inside of a guest OS.  %Thus, the current
%manner of constructing OS VMs does not provide strong resilience
%Yet, when running it is possible to exploited in a number of VMs, indicating
%that isolation alone is not sufficient protection.
In this paper, we use a new insight about where security bugs lie to devise a design to improve the security of applications
run in OS VMs.
  %document the development of a different type of virtual
%machine to secure the OS kernel.
We begin by observing that a portion of the OS kernel—the kernel paths accessed
by popular applications in everyday use—contains fewer security bugs than less-used paths. We
leverage this observation to devise the Lock-in-Pop design, which
locks an application, and the POSIX implementation that services it, into
accessing only the well-used popular portion of the kernel.  Using the Lock-in-Pop model, we
implement a virtual machine called Lind.
%a very small trusted computing base. Complex functionalities in Lind are performed
%within a sandbox, using a memory-safe programming language to re-create riskier system calls.
We compare Lind and seven other virtualization systems that were
available at the release of Linux kernel version 3.14.1, and evaluate
their effectiveness in containing the zero-day kernel bugs that have been discovered
since then.
%against zero day
%bugs discovered in the Linux kernel version 3.14.1.
%When Lind was tested against seven other virtual machines to see whether it would
%   trigger any of 35 bugs examined in Linux kernel version 3.14.1, the threat of
%    an attack was reduced to less than 3\%.
Our results show that it is substantially more difficult to trigger zero-day bugs in
Lind (3\%) than existing systems like VirtualBox (40\%), VMWare Workstation
(31\%), Docker (23\%), LXC (34\%), QEMU (14\%), KVM (14\%), and Graphene (23\%).
