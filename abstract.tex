\subsection*{Abstract}
%A possible compromise of privileged code, such as that in the OS kernel, is a
%daunting security threat. Severe vulnerabilities discovered in this code
%have resulted in privilege escalation, denial of service, memory corruption and
%other complications. 
Virtual machines are widely used in practice, in part for their ability to
isolate potentially untrusted code from the rest of the system.
%Virtual machines have been developed to protect kernel code
%through isolation. 
However, it is often possible to trigger zero-day flaws
in the host OS kernel from inside of a guest OS VM.  %Thus, the current 
%manner of constructing OS VMs does not provide strong resilience 
%Yet, when running it is possible to exploited in a number of VMs, indicating
%that isolation alone is not sufficient protection. 
In this paper, we use an observation about where security bugs lie to
devise a new design model that improves the security of applications
run in OS VMs.  %document the development of a different type of virtual
%machine to secure the OS kernel. 
We begin by observing that a portion of the OS kernel (those kernel paths used 
by popular applications in everyday use) contain fewer security bugs. We 
leverage this knowledge to devise a novel design called Lock-in-Pop, which 
locks an application (and the POSIX implementation that services it) into only 
accessing that portion of the kernel.  Using the Lock-in-Pop model, we 
implement a virtual machine called Lind.
%a very small trusted computing base. Complex functionalities in Lind are performed
%within a sandbox, using a memory-safe programming language to re-create riskier system calls.
We took the versions of Lind and seven other virtualization systems that were
available at the release of Linux kernel version 3.14.1 and evaluated
their effectiveness containing the zero day kernel bugs that were discovered 
since then.
%against zero day
%bugs discovered in the Linux kernel version 3.14.1.
%When Lind was tested against seven other virtual machines to see whether it would
%   trigger any of 35 bugs examined in Linux kernel version 3.14.1, the threat of
%    an attack was reduced to less than 3\%. 
Our results show it is substantially harder to trigger zero day bugs in 
Lind (3\%) than existing systems like VirtualBox (40\%), VMWare Workstation
(31\%), Docker (23\%), LXC (34\%), QEMU (14\%), KVM (14\%), and Graphene (23\%).
