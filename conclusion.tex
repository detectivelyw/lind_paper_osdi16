\section{Conclusion}
\label{sec.conclusion}

In this paper, we propose a new security metric based on quantitative measures derived from
the execution of kernel code when running user applications.
We devise a new design for secure virtual machines, called \emph{Lock-in-Pop}, 
as it locks out application access to all kernel code except that found in paths frequently used by 
popular programs that contain fewer vulnerabilities. 
We implement a prototype system Lind, with a minimized TCB and interacting with the kernel in only popular paths. 
Lind addresses the need to support risky system calls by securely reconstructing complex, yet essential OS functionality 
inside a sandbox. 
Evaluation results have shown that Lind is the most effective system in preventing zero-day Linux kernel bugs, 
when compared to seven other virtualization systems, such as VirtualBox, VMWare Workstation, Docker, LXC, 
QEMU, KVM and Graphene.

All of the data and source code for this paper is available at the Lind website~\cite{Lind}. 