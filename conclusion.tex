\section{Conclusion}
\label{sec.conclusion}

In this paper, we propose a new security metric based on quantitative measures derived from
the execution of kernel code when running user applications.
After determining that fewer bugs exist in popular paths associated with frequently-used
programs, we devise a new design for secure virtual machines, called \emph{Lock-in-Pop}.
As the name implies, the design scheme locks out access to all
kernel code except that found in paths frequently used by
popular programs. We test the Lock-in-Pop idea by implementing a prototype system
called Lind, that features a minimized TCB and prevents direct access to application
calls from less-used, risker paths.
Instead, Lind support complex system calls by securely re-creating
essential OS functionality inside a sandbox.
In tests against VirtualBox, VMWare Workstation, Docker, LXC,
QEMU, KVM and Graphene, Lind emerged as the most effective system in preventing
zero-day Linux kernel bugs.
All of the data and source code for this paper is available at the Lind website~\cite{Lind}.
